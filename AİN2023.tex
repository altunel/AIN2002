\documentclass{article}

\documentclass[pdftex,12pt,a4paper]{article}

\usepackage{graphicx}  
\usepackage[margin=2.5cm]{geometry}
\usepackage{breakcites}
\usepackage{indentfirst}
\usepackage{pgfgantt}
\usepackage{pdflscape}
\usepackage{float}
\usepackage{epsfig}
\usepackage{epstopdf}
\usepackage[cmex10]{amsmath}
\usepackage{stfloats}
\usepackage{multirow}
\usepackage{minted}
\usepackage{xcolor}
\usepackage{adjustbox}
\usepackage[utf8]{inputenc} % allow utf-8 input
\usepackage[T1]{fontenc}    % use 8-bit T1 fonts
\usepackage{hyperref}       % hyperlinks
\usepackage{url}            % simple URL typesetting
\usepackage{booktabs}       % professional-quality tables
\usepackage{amsfonts}       % blackboard math symbols
\usepackage{nicefrac}       % compact symbols for 1/2, etc.
\usepackage{microtype}      % microtypography
\usepackage{xcolor}         % colors

\renewcommand{\refname}{REFERENCES}
\linespread{1.3}

\usepackage{mathtools}
%\newcommand{\HRule}{\rule{\linewidth}{0.5mm}}
\thispagestyle{empty}
\begin{document}
\begin{titlepage}
\Large{
\begin{tabular}{rcl}
2004287  & : & Mert \ ACAR \\
2004099  & : & Celal Berkay \ ALTUNEL\\
2101196  & : & Mohammed Yazan \ KURDİ He made no contribution \\ 
         &  & to the PROJECT \\ \\
\end{tabular}}
\end{table}
        Mert \ ACAR: \href{https://github.com/AaCcRr/AIN}{https://github.com/AaCcRr/AIN} \\
        \ Celal Berkay \ ALTUNEL: \href{https://github.com/AaCcRr/AIN}{https://github.com/AaCcRr/AIN}\\
\vspace{2.8cm}
\textbf{\Large{2023}}




\section{Analysis}

We analyzed our data sequentially with the purpose of obtaining our data in the most accurate form and ensuring that our model, which we will create, produces reliable results. We selected various methods of analysis based on the available data. We calculated the results using three different approaches: t-test, correlation, and percentage analysis.

\subsection{Calculate the correlation between avg glucose level and stroke}

\begin{itemize}
    \item The correlation between "avg_glucose_level" and "stroke" is calculated as 0.15. This\\ value indicates a positive relationship between the two variables. However, since the \\ correlation coefficient is 0.15, this relationship is considered weak. In other words, \\ the correlation between "avg_glucose_level" and "stroke" is not very strong.\\

    \item The correlation between "bmi" and "stroke" is calculated as 0.07. This value \\ indicates a very weak relationship between the two variables. With a correlation \\ coefficient of 0.07, the relationship between "bmi" and "stroke" is almost negligible.\\    
    \item The correlation between "age" and "stroke" is calculated as 0.26. This value\\ indicates a moderate positive relationship between the two variables. With a correlation\\ coefficient of 0.26, the relationship between "age" and "stroke" is moderately positive. \\
\end{itemize}

These correlation values quantify the relationship between the variables. However, it's\\important to note that correlation measures only linear relationships and may disregard the \\ influence of other factors. Therefore, it is important to consider other analysis methods and \\specific characteristics of the data as well.


\subsection{T test}

\label{gen_inst}
\begin{itemize}

\item The p-value of 0.9673092270304077 suggests that there is no statistically significant difference between the Rural and Urban areas. This means that the difference observed in the data is likely due to chance, and there is no strong evidence to conclude that there is a meaningful difference between the two regions.

\item The p-value of 0.06459348879850953 indicates that there is no statistically significant difference between the male and female areas. This suggests that any observed differences in the data may be attributed to random variation, and there is not enough evidence to support the presence of a significant difference between the two genders.

\item The extremely small p-value of 1.8922577018821537e-45 indicates that there is a statistically significant difference between the Yes and No areas. This means that the observed difference in the data is highly unlikely to occur by chance alone. The results provide strong evidence to support the presence of a meaningful difference between the two categories.

In statistical analysis, the p-value helps determine the significance of the observed differences or relationships. A p-value less than a predetermined significance level (commonly 0.05) is typically considered statistically significant, suggesting that the observed difference is unlikely due to random chance. On the other hand, a p-value greater than the significance level indicates that there is not enough evidence to support the presence of a significant difference or relationship.

\end{itemize}

\subsection{Analysis by Percentage}
\label{headings}
\begin{itemize}

\item "Residence type'e göre stroke yüzdeleri" means "Stroke percentages by Residence_type".\\ The output shows the stroke percentages for different residence types (Rural and Urban).\\ According to the results, the stroke percentage is approximately \%4.14 for Rural areas and\\ \%4.12 for Urban areas. This suggests that there is a slight difference in stroke prevalence\\ based on the residence type, but the difference is not substantial.\\

\item "smoking status'e göre stroke yüzdeleri" means "Stroke percentages by smoking_status".\\ The output displays the stroke percentages for different smoking statuses (Unknown, formerly\\ smoked, never smoked, and smokes). The results show that the stroke percentage is\\ approximately\\ \%2.38 for Unknown smoking status, \%6.80 for formerly smoked, \%4.09 for never\\ smoked, and \%5.04 for smokes. This indicates that there are variations in stroke\\ prevalence  based on smoking status, with higher percentages observed for formerly smoked and\\ smokes categories compared to other categories.

\item Overall, these analyses provide insights into the relationship between \\ and stroke prevalence. They help identify potential associations between variables and stroke occurrence,\\ which can be useful for further investigation or predictive modeling.

\end{itemize}

Overall, these analyses provide insights into the relationship between different variables and\\ stroke prevalence. They help identify potential associations between variables and stroke\\ occurrence, which can be useful for further investigation or predictive modeling.\\

\section{Model}

\subsection{One-Hot Encoding}

One-Hot Encoding is a method for converting categorical variables to numerical representations. In this method, binary columns (dummy variables) are created for each category. The One-Hot Encoding processes used in the related data set are as follows:

\begin{item}
   One-Hot Encoding for the \item "gender" column:
     \begin{lstlisting}[language=Python]
     pd.get_dummies(data["gender"])
     \end{lstlisting}
     This operation represents the "Male" and "Female" categories in the "gender" column with binary columns.

   \item One-Hot Encoding is applied similarly for other categorical variables.

\end{itemize}

\subsection{Deleting Original Columns}

After One-Hot Encoding is complete, the encoded categorical columns are extracted from the original dataset. This is done because the coded columns already contain the information of the original categorical variables.

\subsection{Creating the Updated DataFrame}

An empty DataFrame is created to store the updated data and the corresponding data rows for each ID are pulled from the original dataset and added to this DataFrame.

\subsection{Removing the ID Column}

After the rows for each ID are combined, the ID column is removed. This means that the ID will not be used as an input to the model.

\subsection{Preparing Input and Output Data}

Input and output data are parsed from the updated DataFrame. Input data is all columns except the "stroke" column. The output data is the "stroke" column.

\subsection{Creating the Model}

A sequential model is defined in the section where the model is created. This model consists of dense layers and is configured with different activation functions.

\subsection{Compilation of Model}

The model is compiled by specifying the optimizer, loss function, and metrics. In this step, it is determined how the model will be learned and how its performance will be measured.

\subsection{Training the Model}

The model is trained in a specified number of epochs and mini-groups with the specified input and output data. In this step, the model improves its performance by learning the patterns on the data.

\subsection{Result}

The One-Hot Encoding process converts categorical variables to numerical representations, enabling machine learning models to process data more effectively. In this document, I've covered how One-Hot Encoding is implemented and other steps in the data preparation process.
\section{Division of Labor}
\subsection{Celal Berkay Altunel:}
    Data analysis and simplification.
\subsection{Mert Acar:}
    Artificial intelligence modelings
\section*{References}


References follow the acknowledgments in the camera-ready paper. Use unnumbered first-level heading for
the references. Any choice of citation style is acceptable as long as you are
consistent. It is permissible to reduce the font size to \verb+small+ (9 point)
when listing the references.
Note that the Reference section does not count towards the page limit.
\medskip


{
\small

[1] Alexander, J.A.\ \& Mozer, M.C.\ (1995) Template-based algorithms for
connectionist rule extraction. In G.\ Tesauro, D.S.\ Touretzky and T.K.\ Leen
(eds.), {\it Advances in Neural Information Processing Systems 7},
pp.\ 609--616. Cambridge, MA: MIT Press.


[2] Bower, J.M.\ \& Beeman, D.\ (1995) {\it The Book of GENESIS: Exploring
  Realistic Neural Models with the GEneral NEural SImulation System.}  New York:
TELOS/Springer--Verlag.


[3] Hasselmo, M.E., Schnell, E.\ \& Barkai, E.\ (1995) Dynamics of learning and
recall at excitatory recurrent synapses and cholinergic modulation in rat
hippocampal region CA3. {\it Journal of Neuroscience} {\bf 15}(7):5249-5262.
}

%%%%%%%%%%%%%%%%%%%%%%%%%%%%%%%%%%%%%%%%%%%%%%%%%%%%%%%%%%%%

\end{document}
